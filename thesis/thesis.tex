\documentclass[twoside,11pt,openright]{report}

\usepackage[latin1]{inputenc}
\usepackage[american]{babel}
\usepackage{a4}
\usepackage{latexsym}
\usepackage{amssymb}
\usepackage{amsmath}
\usepackage{epsfig}
\usepackage[T1]{fontenc}
\usepackage{lmodern}
\usepackage[labeled]{multibib}
\usepackage{color}
\usepackage{datetime}
\usepackage{epstopdf} 

\renewcommand*\ttdefault{txtt}

\newcommand{\todo}[1]{{\color[rgb]{.5,0,0}\textbf{$\blacktriangleright$#1$\blacktriangleleft$}}}

\newcites{A,B}{Primary Bibliography,Secondary Bibliography}

% see http://imf.au.dk/system/latex/bog/

\begin{document}

%%%%%%%%%%%%%%%%%%%%%%%%%%%%%%%%%%%%%%%%%%%%%%%%%%%%%%%%%%%%%%%%%%%%%%%

\pagestyle{empty} 
\pagenumbering{roman} 
\vspace*{\fill}\noindent{\rule{\linewidth}{1mm}\\[4ex]
{\Huge\sf Something about Computer Science}\\[2ex]
{\huge\sf Mads Ravn, 20071580}\\[2ex]
\noindent\rule{\linewidth}{1mm}\\[4ex]
\noindent{\Large\sf Master's Thesis, Computer Science\\[1ex] 
\monthname\ \the\year  \\[1ex] Advisor: Hvem\\[15ex]}\\[\fill]}
\epsfig{file=logo.eps}\clearpage

%%%%%%%%%%%%%%%%%%%%%%%%%%%%%%%%%%%%%%%%%%%%%%%%%%%%%%%%%%%%%%%%%%%%%%%

\pagestyle{plain}
\chapter*{Abstract}
\addcontentsline{toc}{chapter}{Abstract}

\todo{in English\dots}

\chapter*{Resum\'e}
\addcontentsline{toc}{chapter}{Resum\'e}

\todo{in Danish\dots}

\chapter*{Acknowledgements}
\addcontentsline{toc}{chapter}{Acknowledgments}

\todo{\dots}

\vspace{2ex}
\begin{flushright}
  \emph{Mads Ravn,}\\
  \emph{Aarhus, \today.}
\end{flushright}

\tableofcontents
\pagenumbering{arabic}
\setcounter{secnumdepth}{2}

%%%%%%%%%%%%%%%%%%%%%%%%%%%%%%%%%%%%%%%%%%%%%%%%%%%%%%%%%%%%%%%%%%%%%%%

\chapter{Introduction}
\label{ch:intro}
In an age where everybody is presented with the possibility of searching through a lot of data, for example ebay.com and bilzonen.dk, we are interested in fast queries with as little overhead space usage as possible. We want to do two-dimensional orthogonal range search in our database, which means we can select two axes on which we can select results between two points on each. We are essentially forming an axis-aligned rectangle and selecting all points which lie within it.



\todo{\dots}

%%%%%%%%%%%%%%%%%%%%%%%%%%%%%%%%%%%%%%%%%%%%%%%%%%%%%%%%%%%%%%%%%%%%%%%
\chapter{Theory}
In this chapter we present the original Orthogonal Range Searching as by Chan et al. and look at how it can be simplified. This simplification serves two purposes: It will be easier to implement which will present a much cleaner code to read and it will be easier to execute since it has much less things to compute and look up.

This simplified range searching is still theoritically faster than the kd-tree which is the standard range search used today. The technique relies heavily on a concept called \texttt{Ball inheritance} which will be explained later.

\section{Original Range Searching}

\section{Simplified Range Searching}

\chapter{\todo{\dots}}
\label{ch:main}

\todo{example of a citation to primary literature: \citeA{lazypropagation2010},
and one to secondary literature: \citeB{ambiguity2010}}

%%%%%%%%%%%%%%%%%%%%%%%%%%%%%%%%%%%%%%%%%%%%%%%%%%%%%%%%%%%%%%%%%%%%%%%

\chapter{Conclusion}
\label{ch:conclusion}

\todo{\dots}

%%%%%%%%%%%%%%%%%%%%%%%%%%%%%%%%%%%%%%%%%%%%%%%%%%%%%%%%%%%%%%%%%%%%%%%

\addcontentsline{toc}{chapter}{Primary Bibliography}
\bibliographystyleA{plain} 
\bibliographyA{refs}
\addcontentsline{toc}{chapter}{Secondary Bibliography}
\bibliographystyleB{plain} 
\bibliographyB{refs} % remove this if you don't need secondary literature

\end{document}

