\documentclass[twoside,11pt,openright]{report}

\usepackage[latin1]{inputenc}
\usepackage[american]{babel}
\usepackage{a4}
\usepackage{latexsym}
\usepackage{amssymb}
\usepackage{amsmath}
\usepackage{epsfig}
\usepackage[T1]{fontenc}
\usepackage{lmodern}
\usepackage[labeled]{multibib}
\usepackage{color}
\usepackage{datetime}
\usepackage{epstopdf} 

\renewcommand*\ttdefault{txtt}

\newcommand{\todo}[1]{{\color[rgb]{.5,0,0}\textbf{$\blacktriangleright$#1$\blacktriangleleft$}}}

\newcites{A,B}{Primary Bibliography,Secondary Bibliography}

% see http://imf.au.dk/system/latex/bog/

\begin{document}

%%%%%%%%%%%%%%%%%%%%%%%%%%%%%%%%%%%%%%%%%%%%%%%%%%%%%%%%%%%%%%%%%%%%%%%

\pagestyle{empty} 
\pagenumbering{roman} 
\vspace*{\fill}\noindent{\rule{\linewidth}{1mm}\\[4ex]
{\Huge\sf Orthogonal Range Search}\\[2ex]
{\huge\sf Mads Ravn, 20071580}\\[2ex]
\noindent\rule{\linewidth}{1mm}\\[4ex]
\noindent{\Large\sf Master's Thesis, Computer Science\\[1ex] 
\monthname\ \the\year  \\[1ex] Advisor: Kasper Green Larsen\\[15ex]}\\[\fill]}
\epsfig{file=logo.eps}\clearpage

%%%%%%%%%%%%%%%%%%%%%%%%%%%%%%%%%%%%%%%%%%%%%%%%%%%%%%%%%%%%%%%%%%%%%%%

\pagestyle{plain}
\chapter*{Abstract}
\addcontentsline{toc}{chapter}{Abstract}

\todo{in English\dots}

\chapter*{Resum\'e}
\addcontentsline{toc}{chapter}{Resum\'e}

\todo{in Danish\dots}

\chapter*{Acknowledgements}
\addcontentsline{toc}{chapter}{Acknowledgments}

\todo{\dots}

\vspace{2ex}
\begin{flushright}
  \emph{Mads Ravn,}\\
  \emph{Aarhus, \today.}
\end{flushright}

\tableofcontents
\pagenumbering{arabic}
\setcounter{secnumdepth}{2}

%%%%%%%%%%%%%%%%%%%%%%%%%%%%%%%%%%%%%%%%%%%%%%%%%%%%%%%%%%%%%%%%%%%%%%%

\chapter{Introduction}
\label{ch:intro}
In an age where everybody is presented with the possibility of searching through a lot of data, for example ebay.com and bilzonen.dk, we are interested in fast queries with as little overhead space usage as possible. We want to do two-dimensional orthogonal range search in our database, which means we can select two axes on which we can select results between two points on each. We are essentially forming an axis-aligned rectangle and selecting all points which lie within it. \todo{Rewrite} \\

\noindent \textbf{Orthogonal Range Searching.} \emph{Orthogonal range searcing} is one of the most fundamental and well-studied problems in computational geometry. Even with extensive research over three decades a lot of questions remain. In this thesis we will focus on $2D$ orthogonal range searching: Given $n$ points from $\mathbb{R}^2$ we want to insert them into a data structure which will be able to efficiently report which $k$ points lie within a given query range $\mathbb{Q} \subseteq \mathbb{R}^2$. This query can be defined as two corners of a rectangle, the lower left corner and the upper right corner, seeing as the query range is orthogonal to the axes. \\

\noindent \textbf{Word RAM Model.} \todo{Skriv om Word RAM Model sammenlignet med de andre typer her}\\

\noindent \textbf{Rank Space Reduction.} Given $n$ points from a universe $U$, the rank of a given point is defined as the amount of points which preceed it in a sorted list. Given two points $a,b \in U: a \leq b$ \emph{iff} $rank(a) \leq rank(b)$. Expanding this concept to 2 dimensions we have a set $P$ of $n$ points on a $U \times U$ grid. We compute for the \emph{x-rank} $r_x$ for each point in $P$ by finding the rank of the x-coordinate amongst all the x-coordinates in $P$. The \emph{y-rank} $r_y$ finds the rank of y-coordinate amongst all of the y-coordinates in $P$. Using \emph{rank space reduction} a new set $P^*$ is constructed where $(x,y) \in P$ is replaced by $(r_x(x), r_y(y) \in P^*$. Given a range query $q = [x_1, x_2] \times [y_1, y_2]$, a point $(x,y) \in P$ is found within $q$ \emph{iff} $(r_x(x), r_y(y))$ is found within $q^* = [r_x(x_1), r_x(x_2)] \times [r_y(y_1), r_y(y_2)]$. \todo{Noget med at deres ordered property remains intact.} Computing the set $P^*$ from $P$ using rank space reduction, $P^*$ is said to be in rank space. While the $n$ points could be represented by $\lg U$ bits in $P$, they can now be represented by $\lg n$ bits in $P^*$ with $\lg n \ll \lg U$ which saves memory. \todo{We have essentially created a mapping between \dots } \\

\noindent \textbf{Ball Inheritance Model.} \\

\noindent \textbf{Outline.} \\

\noindent \textbf{Notation.} lg er base to. epsilon er et meget lille tal \\


\todo{\dots}

%%%%%%%%%%%%%%%%%%%%%%%%%%%%%%%%%%%%%%%%%%%%%%%%%%%%%%%%%%%%%%%%%%%%%%%
\chapter{Theory}

Rank Space Reduction: Given a universe $U$ the grid $U \times U$ contain all the points in coordinate space. In order to preserve space the points are reduced to rank space. 

Ball Inheritance model: Given the rank space reduction we wish to 


In this chapter we present the original Orthogonal Range Searching as by Chan et al. and look at how it can be simplified. This simplification serves two purposes: It will be easier to implement which will present a much cleaner code to read and it will be easier to execute since it has much less things to compute and look up.

This simplified range searching is still theoritically faster than the kd-tree which is the standard range search used today. The technique relies heavily on a concept called \texttt{Ball inheritance} which will be explained later.

\section{Original Range Searching}

\section{Simplified Range Searching}

\chapter{\todo{\dots}}
\label{ch:main}

\todo{example of a citation to primary literature: \citeA{lazypropagation2010},
and one to secondary literature: \citeB{ambiguity2010}}

%%%%%%%%%%%%%%%%%%%%%%%%%%%%%%%%%%%%%%%%%%%%%%%%%%%%%%%%%%%%%%%%%%%%%%%

\chapter{Conclusion}
\label{ch:conclusion}

\todo{\dots}

%%%%%%%%%%%%%%%%%%%%%%%%%%%%%%%%%%%%%%%%%%%%%%%%%%%%%%%%%%%%%%%%%%%%%%%

\addcontentsline{toc}{chapter}{Primary Bibliography}
\bibliographystyleA{plain} 
\bibliographyA{refs}
\addcontentsline{toc}{chapter}{Secondary Bibliography}
\bibliographystyleB{plain} 
\bibliographyB{refs} % remove this if you don't need secondary literature

\end{document}

