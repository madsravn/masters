\chapter{Introduction}
\label{ch:intro}

\noindent \emph{Orthogonal range searching} is one of the most fundamental and well-studied problems in computational geometry. Even with extensive research over three decades a lot of questions remain. In this thesis we will focus on $2D$ orthogonal range searching: Given $n$ points from $\mathbb{R}^2$ we want to insert them into a data structure which will be able to efficiently report which $k$ points lie within a given axis-aligned query rectangle $\mathbb{Q} \subseteq \mathbb{R}^2$. This query can be defined as two corners of a rectangle, the lower left corner and the upper right corner, seeing as the query range is orthogonal to the axes.

The objective of this thesis is to study a variety of \emph{orthogonal range searching} data structures. The main focus will be to introduce the \emph{Simple Range Search} data structure. It is a simplification of an orthogonal range searching data structure by \citet{chanetal}, which will be referred to as the \emph{Original Range Search} data structure. We are going to describe the kd-tree which will be the baseline \todo{andet ord} in our comparison of the Simple Range Search data structure. We are going to describe the range tree which shares some of properties of the Simple Range Search and Original Range Search data structures.

We show that a range query to the Simple Range Search data structure has a faster worst-case running time than a range query to the kd-tree. We also wish to show that the Simple Range Search data structure is able to compete with the kd-tree, and even outperform the kd-tree in some cases. \todo{Mere ...} 

The model of computation used in this thesis is the $w$-bit word-RAM model\todo{cite Fredman} . In the word-RAM model of computation, the memory is divided into words of $w$ bits. Given an array $A[1..n]$ containing $n$ points with coordinates from a universe $U$, a word will have enough bits to store the integer address of any index into $I$ and enough bits to store any element from $U$. Thus, $w = \Omega(\lg n)$ and $w = \Omega(\lg U)$. Under the word-RAM model all standard word operations take constant time. This includes standard word operation from modern programming languages such as integer addition, subtraction, multiplication, division, shifts and the bit-wise operators AND, OR and XOR. Reading a single word from memory or writing a single word to memory also takes constant time. The number of bits in a word is found by the largest element which has to fit into a word. This means that it is often possible to divide the word into smaller logical blocks which can fit more than one integer. \todo{Indivisibility af pointer machine relevant?} \\


\noindent \textbf{Outline.} Has yet to be written. \\

\noindent \textbf{Notation.} The set of integers $\{i, i+1, \dots, j-1, j\}$ is denoted by $[i,j]$. When no base is explicitely given logarithm will have base $2$. $\epsilon$ is an arbitrary small constant greater than $0$. Given an array $A$, $A[i]$ denotes the entry with index $i$ in $A$ and $A[i,j]$ denotes the subarray containing the entries from $i$ to $j$ in $A$. $A[1..n]$ denotes an array $A$ of size $n$ with entries $1$ to $n$. Throughout the thesis the successor of $x$ in a set will be meant as the first number which is greater or equal to $x$ in that set - the same applies for predecessor of $x$. The work will be done under the assumption that no two points will  have the same x-coordinate and no two points will have the same y-coordinate. This is a unrealistic assumption in practice, but it can easily be remedied by having the points lie in a composite-number space since we only need a total ordering of our points.



