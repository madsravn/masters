\chapter{Introduction}
\label{ch:intro}

\todo{Describe primary work}
In this thesis we are going to explore data structure we believe to be faster at orthogonal range search than a kd-tree. We are going to introduce the a range reporting data structure by \citet{chanetal}. A simplified version of their data structure will be made and implemented. We wish to show that this simplified version is able to compete with a kd-tree. \\

\noindent \textbf{Orthogonal Range Searching.} \emph{Orthogonal range searcing} is one of the most fundamental and well-studied problems in computational geometry. Even with extensive research over three decades a lot of questions remain. In this thesis we will focus on $2D$ orthogonal range searching: Given $n$ points from $\mathbb{R}^2$ we want to insert them into a data structure which will be able to efficiently report which $k$ points lie within a given query range $\mathbb{Q} \subseteq \mathbb{R}^2$. This query can be defined as two corners of a rectangle, the lower left corner and the upper right corner, seeing as the query range is orthogonal to the axes. \\

\noindent \textbf{RAM, I/O, pointer models} \todo{Skriv om Word RAM Model sammenlignet med de andre typer her. Når vi pakker bits ned i et word kan vi stadig tilgå det hele da access er constant time og operations er constant time}\\

\noindent \textbf{Rank Space Reduction.} Given $n$ points from a universe $U$, the rank of a given point in a sorted list is defined as the amount of points which preceed it in the list. Given two points $a,b \in U: a < b$ \emph{iff} $rank(a) < rank(b)$. Expanding this concept to 2 dimensions we have a set $P$ of $n$ points on a $U \times U$ grid. We compute for the \emph{x-rank} $r_x$ for each point in $P$ by finding the rank of the x-coordinate amongst all the x-coordinates in $P$. The \emph{y-rank} $r_y$ finds the rank of y-coordinate amongst all of the y-coordinates in $P$. Using \emph{rank space reduction} on $P$ a new set $P^*$ is constructed where $(x,y) \in P$ is replaced by $(r_x(x), r_y(y) \in P^*$. Given a range query $q = [x_1, x_2] \times [y_1, y_2]$, a point $(x,y) \in P$ is found within $q$ \emph{iff} $(r_x(x), r_y(y))$ is found within $q^* = [r_x(x_1), r_x(x_2)] \times [r_y(y_1), r_y(y_2)]$. \todo{Noget med at deres ordered property remains intact.} Computing the set $P^*$ from $P$ using rank space reduction, $P^*$ is said to be in rank space. While the $n$ points could be represented by $\lg U$ bits in $P$, they can now be represented by $\lg n$ bits in $P^*$ with $\lg n \ll \lg U$ when $n \ll U$ which saves memory. \todo{We have essentially created a mapping between \dots } \\

%\noindent \textbf{Ball Inheritance Problem.} Given a perfect binary tree with $n$ leaves, we want to distribute $n$ labelled balls which appear in an ordered list at the root to the leaves in $\lg n$ steps. \todo{Rephrase}. The list of balls at a given node will inherited by its children with each child receiving the same amount of balls. Each level of the tree contains the same amount of balls. Eventually each ball reaches a leaf of the tree and each leaf will contain exactly one ball. The identity of a given ball is a node and the index of the ball in that nodes list. The true identity, the information the ball contains, only lies in the leaf. The \emph{ball-inheritance problem} is to track a ball from a given node to a leaf and report the identity of the leaf. \\

\noindent \textbf{Ball Inheritance.} Given a perfect binary tree with $n$ leaves and $n$ labelled balls at the root, the goal is to distribute the balls from the root to the leaves. The balls at the root are contained in an ordered list and for each ball in a node's list, one of its children is picked to inherit the ball such that both children recieve the same amount of balls. The level of a node is defined to be the height of the node from the bottom. The root has the highest level, while each node is one level smaller than its parent. The leaves are at level $0$. Each level of the tree contains the same amount of balls, and at level $i$ each node contains $2^i$ balls. Eventually each ball reaches a leaf of the tree and each leaf will contain exactly one ball. A ball can be identified by a node and the index of the ball in that node's list. Given the identity of a ball at any level, it is possible to follow this identity to its actual coordinates. The goal is to track a balls inheritance from a given node to a leaf and report the identity of the leaf. We call the identity of the leaf, where the actual coordinates reside, the \emph{true identity} of a ball. When refering to ball inheritance, the words \emph{ball} or \emph{point} can be used interchangebly. \\


\noindent \textbf{Outline.} Has yet to be written. \\

\noindent \textbf{Notation.} The set of integers $\{i, i+1, \dots, j-1, j\}$ is denoted by $[i,j]$. When no base is explicitely given logarithm will have base $2$. $\epsilon$ is an arbitrary small constant greater than $0$. Given an array $A$, $A[i]$ denotes the entry with index $i$ in $A$ and $A[i,j]$ denotes the subarray containing the entries from $i$ to $j$ in $A$. $A[1..n]$ denotes an array $A$ of size $n$ with entries $1$ to $n$. Throughout the thesis the successor of $x$ in a set will be meant as the first number which is greater or equal to $x$ in that set - the same applies for predecessor of $x$. The work will be done under the assumption that no two points will  have the same x-coordinate and no two points will have the same y-coordinate. This is a unrealistic assumption in practice, but it can easily be remedied by having the points lie in a \emph{composite-number space} since we only need a total ordering of our points. \todo{Remember to mention this later}


